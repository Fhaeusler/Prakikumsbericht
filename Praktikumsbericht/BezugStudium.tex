\chapter{Bezug zum Studium}
% zusammenstellen der Arbeitsmittel Git, slack etc.
Dadurch dass wir in der Hochschule in zahlreichen Projekten bereits schon mit Git, Android Studio und anderen Entwicklungsumgebungen, Slack und Projekt-Management Tools wie Trello gearbeitet haben, brauchte es in diesen Bereichen keine lange Einarbeitungszeit und es konnte sich auf inhaltliche Probleme und dessen Umsetzung konzentriert werden.

%  Projektmanagement
Wie durch die vorherigen Kapitel erkennbar, wurde die Entwicklung mithilfe von Scrum vorangetrieben. Mit dem Project Owner in Kiel, den Scrum Master in Berlin und das Entwicklungsteam zu dem ich auch gehörte. Allerdings gab es keine täglichen Sprints, sondern nur Wöchentliche, da unser Entwicklungsteam nicht Vollzeit und in Deutschland verteilt arbeitete. Dabei half es die einzelnen Rollen zu kennen, sich auf die wöchentlich Sprints vorzubereiten bzw. den Scrum Master bei Problemen rechtzeitig informieren zu können. Damit Probleme bei nachfolgenden Sprints und im Projekt Plan berücksichtigt werden konnten.  %to was half einen dabei der einblick 


% Verteilte systeme  KBE
Ich denke bei meiner Arbeit mit der Rest API, konnte ich am besten aus dem Wissen von Komponentenbasierte Entwicklung und Verteilte System zurückgreifen. Gerade bei der Serialisierung von Objekten mit POJOs oder das OR-Mapping waren mir aus den Vorlesungen bekannt und halfen mir bei der Umsetzung in diesem Projekt. Bei der Rest API hatte ich bereits ein grundlegendes Verständnis, wie die Kommunikation abläuft, wie die Fehlercodes aussehen, was sie bedeuten sowie Anfragen und Antworten ablaufen.
% Mobile Anwendungen Android Studio und ios 

Da ich den Schwerpunkt im Studium  auf Mobile-Anwendungen gelegt habe, hatte ich bereits Grunderfahrungen in der Entwicklung von Smartphone Apps in iOS sowie  Android. Wir hatten uns bei der Entwicklung auf die Sprache Java und gegen Kotlin geeinigt, so musste auch hier kein großes Umlernen erfolgen, da quasi 80 \% des in der Hochschule produzierten Codes Java war. Hier half auch das der Umgang mit Android Studio und deren Struktur, sowie die Paketverwaltung Gradel, deren Problem mir bekannt waren.

% Realtionale Datenbanken
Da wir auch eine Datenbank zum Caching einsetzten, konnte ich vom Wissen aus dem Modul Datenbanken zurückgreifen. Hinsichtlich der Beziehungen zwischen Tabellen und Erstellen von Datenbank abfragen.

Weiterhin wirkte  sich das Umsetzen von kleineren Projekten innerhalb der Kurse aus, womit erlernte Fähigkeiten  verfestigt wurden und wieder schneller ins Gedächtnis gerufen werden.%to weiter positive gedanken zu projekten  

% arbeiten in gruppen 
Neu war der Umgang auch mit fremden Code der interpretiert und verstanden werden musste. Wo man sich vorher zum größten Teil nur mit seinem eigenen Code auseinandersetzen musste, war es nun wichtig zu berücksichtigen das andere diesen Code lesen und verstehen müssen. Hierbei war die Vergabe von Variablen, Klassennamen usw. sowie das Dokumentieren wichtig. Dies wurde im abschließenden Refactoring Prozess sichtbar.







%too verstehen lesen von fremden code 
%tdo nutzen aussagekräftiger klassennamen, variablen und methoden
%tdo frühzeitige kommunikation 