\chapter{Einleitung}
Bereits vor meinem Praktikumsbeginn hab ich vom 1. Dezember 2018 bis 31 Januar 2019 als Werksstudent bei der Intermedcon GmbH gearbeitet. Mir erschien der Zeitraum für ein Praktikum ziemlich kurz, ein neues Projekt im Dezember 2018 Startete und  im letzten Semester besuchte ich nur zwei Kurse.  Da hat es sich  angeboten als Werksstudent in das Team und das neue Projekt einzusteigen, um eine spätere Einarbeitung zu vermeiden und das Projekt von beginn an unterstützen zu können. 
\section{Praktikumsbetrieb}
Die Intermedcon GmbH hat Standorte in Münster, Kiel und Berlin. Ein Großteil der Entwickler befindet sich  zurzeit in der  Dorotheenstraße Berlin, wo auch ich meinen Arbeitsplatz bekommen habe. Ein Umzug in  Größere Büros steht aber schon fest.  Da die Bürokapazitäten noch begrenzt sind, ist den Mitarbeitern freigestellt ihre Arbeit auch im Home Office zu erledigen. Auch wenn der Platz im Büro manchmal etwas enger wurde war ein Konstruktives und gutes Arbeiten möglich. Vor Ort ist  der Technische Leiter zwei Webentwickler und ein  weiterer Android  Entwickler sowie Mitarbeiter von einem Tochterunternehmen. Die Personalabteilung und Geschäftsführer befinden sich in Münster bzw. Kiel. 

Die  Intermedcon GmbH hat sich eHealth Anwendungen spezialisiert, Auftraggeber sind hierbei mit Privaten und Gesetzlichen Krankenkassen sowie Pharma-Unternehmen. Die Dabei  ist Sie  in Zahlreiche öffentlichen Forschungs- und Förderprogrammen vertreten und arbeitet mit  internationalen Partnern zusammen u.a Dänemark, Schweden und der Niederlande. Gerade für die Organisation und Aufbereiten  Internationalen Projekten in Medizineschen Studien werden neue Softwarelösungen benötigt.



