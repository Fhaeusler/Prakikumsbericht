\chapter{Einleitung}\label{einleitung}
Bereits vor meinem Praktikumsbeginn habe ich vom 1. Dezember 2018 bis 31. Januar 2019 als Werksstudent bei der InterMedCon GmbH gearbeitet. Da mir der Zeitraum für ein Praktikum ziemlich kurz erschien, ein neues Projekt im Dezember 2018 startete und ich im letzten Semester nur zwei Kurse besuchte. Hat es sich angeboten als Werksstudent in das Team und das neue Projekt einzusteigen, um eine spätere Einarbeitung zu vermeiden und das Projekt von Beginn an unterstützen zu können. 
\section{Praktikumsbetrieb}
Die InterMedCon GmbH hat Standorte in Münster, Kiel und Berlin. Ein Großteil der Entwickler befinden sich zurzeit in der Dorotheenstraße Berlin, wo auch ich meinen Arbeitsplatz bekommen habe. Ein Umzug in größere Büros steht aber schon fest. Da die Bürokapazitäten noch begrenzt sind, ist den Mitarbeitern freigestellt ihre Arbeit auch im Home Office zu erledigen. Auch wenn der Platz im Büro manchmal etwas enger wurde, war ein konstruktives und gutes Arbeiten möglich. Vor Ort ist der Technische Leiter, zwei Webentwickler und ein weiterer Android Entwickler sowie Mitarbeiter von einem Tochterunternehmen. Die Personalabteilung und Geschäftsführung befinden sich in Münster bzw. Kiel. 

Die InterMedCon GmbH hat sich auf eHealth Anwendungen spezialisiert, der Kundenstamm umfasst Private und Gesetzliche Krankenkassen sowie Pharma-Unternehmen. Dabei ist die InterMedCon GmbH in zahlreichen öffentlichen Forschungs- und Förderprogrammen vertreten und arbeitet mit internationalen Partnern zusammen u.a in Dänemark, Schweden und der Niederlande. Gerade für die Organisation und Aufbereitung Internationaler Projekte in Medizinischen Studien werden neue Softwarelösungen benötigt. 




