\chapter{Einleitung}
\section{Praktikumsbetrieb}
Bereits vor meinem Praktikumsbeginn hab ich vom 1. Dezember 2018 bis 31 Januar 2019 als Werksstudent bei der Intermedcon GmbH gearbeitet. Mir erschien der Zeitraum für ein Praktikum ziemlich kurz, ein neues Projekt im Dezember 2018 Startete und  im letzten Semester besuchte ich nur zwei Kurse.  Da hat es sich  angeboten als Werksstudent in das Team und das neue Projekt einzusteigen, um die Einarbeitung zu erleichtern. 

Die Intermedcon GmbH hat Standorte in Münster, Kiel und Berlin. Ein Großteil der Entwickler befindet sich in Berlin Dorotheenstraße, wo ich auch meinen Platz bekommen habe. Vor Ort ist auch der Technische Leiter zwei Webentwickler und ein  Android  Entwickler sowie Mitarbeiter von einem Tochterunternehmen. Den Mitarbeitern wurde es freigestellt ihre Arbeit auch im Home Office zu erledigen. Die Personalabteilung und Geschäftsführer befinden sich in Münster bzw. Kiel. 

Das Unternehmen hat sich spezialisiert auf  eHealth Anwendungen, Auftraggeber sind hierbei mit Privaten und Gesetzlichen Krankenkassen sowie Pharma-Unternehmen. Die Intermedcon GmbH ist hierbei in Zahlreiche öffentlichen Forschungs- und Förderprogrammen vertreten und arbeitet mit Zahlreichen internationalen Partnern zusammen u.a Dänemark, Schweden und Niederlande. Gerade für die Organisation und Aufbereiten von Medizineschen Studien werden neue Softwarelösungen benötigt, gerade in Internationalen Projekten.
%todo test entfernen
\gls{computer}
\ac{ts}



