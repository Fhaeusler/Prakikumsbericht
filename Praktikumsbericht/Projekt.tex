\chapter{Projekt}
\section{Projektbeschreibung iSpine}\label{beschreibung}
Beim \glqq iSpine\grqq handelt es sich um ein von der Europäischen Union gefördertes Projekt, welches im Verbund mit den Dänischen Partnern Sens Motion durchgeführt wurde. Es sollte eine Intelligente Selbstüberwachung für eine effiziente ambulante Behandlung von ambulanter Rückschmerzen entwickelt werden. Dafür sollte von der Firma Sens Motion ein neuer Sensor entwickelt werden, der verschiedene Bewegungsarten erkennen kann. Weiter sollte eine App zur Motivation zur Unterstützung eines Aktiveren Lebens entwickelt werden, eine Anpassung für weiter Therapien und Krankheitsformen sollte gegeben sein. Die Sensoren sollen mittels Pflaster an den Oberschenkel angebracht werden, Sie sind für den einmaligen Gebrauch und min. 14 Tage Wasserfest. Die Bewegungsarten die ein Sensor ermitteln soll sind: Ruhend, Gehen, Laufen, Fahrradfahren, Schrittzähler.

Die ersten Klinischen Test bei unserem Partner in Kopenhagen waren für Apri 2019 geplant. Hierfür sollten 100 Patienten mit unspezifischen und chronischen Rückenschmerzen ausgesucht werden und sind in 2 Gruppen einzuteilen. Die eine sollte den Sensor und die App erhalten und die 2 nur den Sensor ohne die App. Die Begleitung der Studie war für einen Zeitraum für 6 Wochen angelegt. Die groben Anforderung an die App waren dabei. Sie sollte möglichst leicht Bedienbar sein da auch gerade älter Patienten und Einwanderer die App bedienen sollen. Es sollte die möglichkeit geben Daten wie Alter, Gewicht, Größe zu hinterlegen, um so später Rückschlüsse auf Erkrankungen zu erkennen. Die Daten die von den Sensor von Sens Motion ermittelt wurden sollten in die App integriert werden und der Patient sollte zu diesen Daten Feedback erhalten. Das Aktivitätsniveau sollt er dabei selbst definieren können.

%Im \glqq iSpine\grqq  Projekt soll in einer Studie herausgefunden werden ob sich Patienten über eine Handy-App motivieren lassen sich mehr zu bewegen. Bei den Probanden handelt sich zu meist um Personen die sich nur wenig bewegen. Im ersten Zeitraum der  Studie sollen die Probanden eine gewisse Zeit zugriff auf die Daten erhalten, im zweiten Zeitraum sollen Sie keine Zugriff bekommen. Die Daten werden aber weiterhin aufgezeichnet. Wichtig hierbei ist es dabei herauszufinden ob die Patienten durch die App motiviert werden sich mehr zu bewegen.

%Die Entwicklung erfolgt hierbei in Kooperation mit dem Dänischen Partner \glqq Sens-motion\grqq. Wir die Intermedcon GmbH waren dabei für die Entwicklung der App zuständig, welche Später den Probanden übergeben wurde. So sollen die Daten von einer web-\ac{api} abrufen und für den Patienten ansprechend in einer Android Anwendung visualisiert werden. Sense-motion ist einerseits für die Sensoren zuständig, sowie für die bereitstellen der Daten über die \ac{api}.

\section{Teammitglieder und deren Aufgaben}\label{team}
Beim iSpine Projekt handelt sich um die erst native Android-App die von der Intermedcon GmbH entwickelt wurde. Die Zusammenstellung des  Team erfolgte erst kurz vor dem Start des Projektes, im November. Hierbei hat der Geschäftsführer das Team zusammengestellt und später ein Teil der Koordination mit den Dänischen Partnern durchgeführt. Die Projektführung wurde zum großen teil von unserem Technischenleiter aus Berlin und wurde später durch eine Projektleiterin aus Kiel unterstützt, da  die Projekt-Managerin erst später im Unternehmen eingestellt wurde. Weiter waren  zwei Werksstudenten /-in als Android Entwickler sowie ein Designerin für die Mockups eingestellt worden. Die ihren Sitz nicht in Berlin hatten und ihre Arbeit im Homeoffice verrichteten. Ab Januar ist ein weiterer Android Entwickler, erst als Teilzeit und später als Vollzeit zum Team in Berlin hinzugekommen. Wie oben in der Einleitung bereits erwähnt, habe ich erst als Werksstudent und ab Februar im Praktikum als Android-Entwickler das Team unterstützen dürfen.
\section{Projektablauf}
\subsection{Beginn/Vorbereitung}\label{kapBegin}
Projektstart für die App-Entwicklung  war der 3.Dezember.Das Gesamtprojekt Startete aber bereits weit vor diesem Datum, mit der Entwicklung des Sensor und der \ac{api} für den zugriff auf Daten durch die Dänischen Partner. Es wurden bereits Studien geplant und erste Auswertungen mit der Hilfe einer Web-App durchgeführt. Zielgruppen wurden analysiert und grobe Anforderungen spezifiziert.

Wie bereits oben erwähnt Startete am 3. Dezember mit einer Kickoff Veranstaltung in Form einer Telefonkonferenz mit den Dänischer Partnern und der Projektleitung die Entwicklung der App. Ich durfte auch an diesem Gespräch teilnehmen, hierbei wurde uns die \ac{api}-Schnittstelle übergeben und uns erste Technischer Erläuterungen über deren Funktionalitäten. Weiter wurden Technische Anforderungen  und eine grobe Zeitplanung besprochen. Dabei war gerade aus unser Position eine konkrete aussage zu treffen sehr schwer. Da viele Faktoren unklar waren, wie zum Beispiel Personalstärke, Fähigkeiten und Funktionsweise der \ac{api}, Auswertung der Daten. %%Zum Anfang stand auch noch zur diskussion dass wir die Auswertunf der Daten machen.
Hierbei wurde auch entschieden dass wir nicht die Daten des Sensor zum Server hochladen, sondern dies durch eine bereits vorhandene App erledigt werden soll. Um durch den begrenzten Zeitraum bis zum Studienbeginn zu erleichtern. Ein erster Prototyp sollte bereits Mitte bis Ende Januar fertig sein und die erste Studie sollt dann ende Februar beginnen.

Am darauffolgenden Mittwoch fand die erst Telefonkonferenz zwischen dem in Kapitel \ref{team} beschrieben Entwickler-Team statt. Hierbei wurden die Inhalte der Telefonkonferenz mit den Dänischen Partnern besprochen.% sowie erste Aufgaben besprochen. 
Weiterführend musste sich auf die einzusetzenden Arbeitsmittel geeinigt werden, da das Team an mehreren Standorten ihre Arbeitete verrichtete.

Nachfolgend wurde die Arbeit mit folgenden Arbeitsmitteln unterstützt:
\begin{description}
	\item[Slack] zur Kommunikation innerhalb des Teams, als auch mit den Dänischen Partnern.
	\item[\gls{git}] als Versionsverwaltungssoftware, was das Arbeiten am selben Code sehr vereinfacht.
	\item[Tiga] als Projektmanagementsoftware, wo jeder seine Aufgaben zugeteilt bekommen hat und der Arbeitsstand festgehalten wurde. Später konnte hier auch ein Bug-Report eingerichtet werden, wo die Dänischen Partner Fehler hinterlegen konnten.
	\item[GoTo-Meeting] Für die Wöchentlichen Sprints die als  Telefonkonferenzen durchgeführt wurden.
	\item[Android-Studio] Eine von Google und IntelliJ  basierende Entwicklungsumgebung.
\end{description}
 Es wurde sich auch auf eine Wöchentlich Iteration geeinigt, wo jeder seine Arbeitsfortschritte und  Probleme im Arbeitsprozess vortragen  konnte.
 
\subsection{Durchführung}
Wie bereits erwähnt war unser App abhängig von den Daten die wir von der API bekommen. Daher mussten einige Punkte im Vorfeld geprüft werden:
\begin{itemize}\label{stichpunkte}
	\item Wie erfolgt die Authentifizierung?
	\item Wie und wer legt neue Patienten an?
	\item Welche Daten Stehen uns in der API zur Verfügung?
	\item Welche Daten brauchen wir?
	\item Welche Request müssen gesendet werden, um an die gewünschten Daten zu kommen?
\end{itemize}
Hierbei kam es gerade in der Anfangszeit zu Problemen. Da es sich bei der API um eine Beta Version handelte, weitere Änderungen vorgenommen wurden und die Verfügbarkeit eingeschränkt war.
Wir haben dafür eine kurze API Dokumentation bekommen und Testanfragen über "\gls{postman}" realisiert, um das verhalten der API zu Analysieren. Dabei war eine funktionierende Kommunikation mit den Dänischen Partnern wichtig! 
Nachdem die Fragen geklärt waren konnten erste Mockups und User Sorys erstellt werden. In der APP-Entwicklung konnten dann die ersten Anmeldeseiten erstellt werden, sowie sich gedanken   über einzusetzende Bibliotheken die eine Anfrage und Verarbeitung der Daten von die API Erleichtern sollten.
Die unzuverlässige funktionierende API und neue Anforderungen machte ein lokales Zwischenspeichern der Daten unverzichtbar. 
Damit war das vorhergehende Speichermodell nicht mehr möglich/nötig und es musste auf ein lokale Datenbank umgestellt werden. Weiter Anforderungen wie eine Hintergrundaktualisierung oder Notifikation machte den Einsatz von "LiveData"  (Observer) Notwendig. Dabei wurden  Daten die durch die Api in der Datenbank Verändert wurden,  in der aktuellen Ansicht aktualisiert. Es sollten auch eigene Daten mit Hilfe der API hochgeladen werden. Damit die Nutze bei wechsel ihres Gerätes nicht ständig ihre Persönlichen Daten neu eingeben müssen.
%todo Weiter Beschreibungen
\subsection{Abschluss}\label{abschluss}
Zwei bis drei Wochen vor dem Start der Studie, bekamen die Kliniker die App ausgehändigt um erste Tests durchzuführen. In einer Wöchentlichen Telefonkonferenz wurden kleine Änderungen,Erweiterungen oder  Probleme besprochen. Die wie bereits beschrieben im Bug-Report festgehalten  und gegebenenfalls in die App intrigiert oder behoben wurden. Hierbei kam es auch  zur Anpassungen an der API, so dass der Anmeldeprozess umgestellt wurde, von Nutzernamen und Password auf Telefon und SMS/Pin. Durch den Einsatz der Bibliotheken war die Umstellung nicht so ein großes Problem wie erst vermutet. Es mussten nur wenige Klassen und UI-Elemente angepasst werden. Weiter musste getestet werden ob alle "Notifiakation" funktionieren, ob die Hintergrundsynchronisation mit der externen App funktioniert. Da die Studie in Dänemark durchgeführt wurde, musst  alle UI-Elemente Übersetzt und deren konkrete Übertragung getestet werden. Da in der Studie Probanden auch ihre eigenen Geräte nutzen durften, war es wichtig die App auch für Verschiedene Displaygrößen auszurichten. Ein großer Teil der Studie nutzte allerdings bereitgestellte Smartphones, diese Geräte mussten beschafft Konfiguriert und mit einer \ac{mdm}%todo was ist mdm
ausgestattet werden. Damit spätere Updates oder Erweiterungen schneller eingespielt werden können. Hierbei wurden alle erkannten Bugs erfolgreich behoben und die Studie konnte lediglich durch eine kurze Verzögerung durch die Lieferung der Smartphones gestartet werden. Während der Studienphase gab es weitere Technischem Gespräche mit den Dänischen Partnern. Dabei kamen lediglich nur noch kleinere Fehler zum Vorschein, die schnell behoben werden konnten, uns wurde ansonsten ein erfolgreichen Start in den ersten zwei Wochen der Studie signalisiert wurde. 

\section{Aufgaben am Projekt}
Wie in Kapitel \ref{kapBegin} durfte ich an der ersten Kickoff Veranstaltung mit den Dänischen Partnern teilnehmen. Um auch erste technische Fragen bezüglich der einzusetzenden API stellen zu können und das Dänische Team kennenzulernen, da die Umsetzung eine enge Zusammenarbeit voraussetzt. Dies war wichtig da ich nachfolgend im Team für die Datenbeschaffung über die API Verantwortlich sein durfte.

Dabei war es erst mal notwendig den Request Ablauf der API herauszufinden, um die in Kapitel \ref{stichpunkte} gestellten fragen beantworten zu können. Dies ging einerseits über eine kleine Webdokumentation und da nicht alle endpunkt richtig beschrieben waren über \gls{postman}. Hiermit konnte erstmals außerhalb der Anwendung das Verhalten der API Analysiert werden. Die gewünschten Request anfragen gefunden und die ersten Daten empfangen werden. Nachfolgend sollte ich eine Lösung finden, die Daten von der API in unsere App zu bekommen. Hierbei hat sich die Bibliothek "\gls{retrofit}" sehr nützlich herausgestellt. Bei \gls{retrofit} werden die Anfragen in einem Inteface beschrieben wie header, Anfrage (POST, GET), Response Objekt , Request Objekte und Endpunkt der API und Retrofit baut mit der Hilfe der Basisadresse die Anfrage zusammen und liefert die Antwort wieder zurück. Diese Request und Respose Objekte werden als \ac{pojo} Objekte beschrieben und die Bibliothek kümmert sich um die richtige Serialisierung. Ein manuelles parsing der JSON objekte ist somit nicht mehr notwendig, dies hat sich gerade bei Umstellungen bei API Anfragen oder bei Veränderungen von Objekten vorteilhaft erwiesen.

Da sich die API im verlauf des Projekte als nicht zuverlässig herausgestellte  und wir zum vorantreiben und testen der UI-Elemente sowie der weiteren Logik auf Daten angewiesen waren, haben wir uns für ein caching der Daten entschieden. Ich durfte mir darübre gedanken machen und entschieden uns für das Caching  auf die \gls{room} Bibliothek. Es handelt sich dabei um eine Relationale Datenbank und man kann die von der API bereits verwendeten \ac{pojo} Objekte für das \ac{orm} wiederverwenden. Die Bibliothek macht den zugriff auf die Datenbank einfach über ein \ac{dao}  möglich und übernimmt ebenfalls das \ac{or}-Mapping. Bei Speicherung von nicht Primitive Objekten  gibt es eine einfach Möglichkeit Konverter zu schreiben. 
Durch den Umstieg auf die Room Datenbank wurde das vorhandene Speicherkonzept verworfen und die Funktionen in die neue Datenbankstruktur migriert.
%Da nicht mit einem so umfangreichen Speicherung von Daten gerechnet wurde, existierte bereits eine SQL-Lite Datenbank, in der Persönliche Daten und andere Werte hinterlegt wurden. Um nicht mehrgleisig zu fahren wurden diese Daten in die Room-Datenbank integriert.%todo erstellen eines ERM und erstellen eines kleinen Klassen Diagramms zugriff auf API und Room Datenbank

Wie in \ref{abschluss} beschrieben kamen es durch Veränderungen an der API noch zu weiteren Anpassungen innerhalb unserer App. So wie erwähnt sollte der Authentifizierung prozess geänderte weden, es sollten eigegen Metadaten zum Server gesendet werden, wir haben weitere Metadaten erhalten. So dass wir auch neue Funktionoalitäten integrieren sollten. So konnte die in \ref{beschreibung} beschriebene Feedback periode hinzugefügt und die persönlichen Daten auf dem Server gespeichert werden.
%Da sich die \ac{api} noch in einer Beta Version befand kam es noch zu Veränderungen, auf die wir im Entwicklungsprozess reagieren mussten. So wurde der Authentifizierungsprozess mit Anmeldenamen + Passwort auf Telefonnummer + SMS/Pin umgestellt. Weiter sollten Metadaten auf dem Server gespeichert werden oder es sollte ermöglicht werden den Patienten eine Feedback-Zeitraum zuordnen zu können. In dem er seine Bewegungsdaten sehen kann ansonsten sind  Sie verborgen.


Weitere Anforderungen und der Einsatz von Hintergrundaktualisierungen machte weiter Veränderungen notwendig. Hierbei durfte ich mich mit dem Einsatz von LiveData beschäfitgen. LiveData ist ein Observer-Muster, es ermöglichte  Liefecycle Konflikte zu verhindern und das Zusammenspiel zwischen API und Room-Datenbank zu verbessern. Da man die Tabellen einer Datenbank direkt als Oberverable Objekt deklarieren kann, werden Veränderungen die durch die  API hervorgerufen werden direkt  an die aktuelle View weitergegeben.

Zum ende des Projektes durfte ich  noch bei paar Frontend arbeiten unterstützen. Dabei ging es meistens um responsive Anpassungen zwischen den verschiedenen Display Größen(Smartphone,Tablet), kleinere Fehlerbehebungen und das Implementieren weiterer Übungen in Form von Bildern.%todo siehe screanshots  
Final wurden Smartphones beschafft, die Installiert, Konfiguriert und Versendet werden mussten. Dies Stellte den Abschluss des Projektes da.



% Einarbeitung in die Api

% lokales speichern von deaten da Api  unzuverlässige 

% darusfolgende einrichtung einer lokalen relationalen Datenbank

%  später änderungen an der Api (LogIn prozess) metadaten hochladen feedbachtime

% weiter anforderungen machte liveData notwendig . Gerade im zusammenspiel Datenbank und Api

% Anpassungen an Tablet 